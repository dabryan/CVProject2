
%This is the hello world document, it is very usefull
%As you can see the character % is used for comments

\documentclass[a4paper,12pt]{article}
%\documentclass[doc]{apa}
\usepackage[usenames,dvipsnames]{color}
\usepackage{graphicx}
\usepackage{multicol}
\usepackage{multirow}
\usepackage{amsmath}
%\setlength\parindent{0pt}
%\usepackage{figure}
%\usepackage[margin=0.5in]{geometry}

\begin{document}
	
\begin{itemize}
    \item Normalize image points
        \begin{itemize}
            \item \textbf{Centroid is at the origin}. We create the matrix $T_{trans}$ for each
                camera like this:
                \begin{equation}
                    \left [
                    \begin{array}[ ]{c c c}
                        1 & 0 & -\mu_x\\
                        0 & 1 & -\mu_y\\
                        0 & 0 & 1\\
                    \end{array}
                \right ]
                \end{equation}
                And we multiply each point of the cameras to they corresponding $T$ matrix like this:
                $Tx_i$.
            \item \textbf{RMS distance from the origin is $\sqrt{2}$.} First compute the RMS of the
                available points:
                \begin{equation}
                    \sqrt{ \frac{1}{n} \sum_{i=1}^n \left(  (x_i - \mu_x)^2 + (y_i- \mu_y)^2) \right) }
                \end{equation}
                Then create $T_{scale}$ and multiply it to each point in the camera. T is:
                \begin{equation}
                    T_s = \left[ 
                    \begin{array}[]{ccc}
                        \sqrt{2}/RMS & 0 & 0 \\
                        0 & \sqrt{2}/RMS  & 0 \\
                        0 & 0 & 1 
                    \end{array}
                     \right]
                \end{equation}
        \end{itemize}
    \item \textbf{Multiply each point by $T_n = T_sT_t$ like this $[u v 1]' = T_nx$. Do it for each camera.}
    \item 
\end{itemize}
\end{document}
